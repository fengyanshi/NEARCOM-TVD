\documentclass[11pt]{article}

\usepackage{times}
\usepackage{amsbsy}

\usepackage{geometry}                % See geometry.pdf to learn the layout options. There are lots.
\geometry{letterpaper}                   % ... or a4paper or a5paper or ... 
%\geometry{landscape}                % Activate for for rotated page geometry
%\usepackage[parfill]{parskip}    % Activate to begin paragraphs with an empty line rather than an indent
\usepackage{graphicx}
\usepackage{amssymb}
\usepackage{epstopdf}

\DeclareGraphicsRule{.tif}{png}{.png}{`convert #1 `dirname #1`/`basename #1 .tif`.png}

\newcommand{\be}{\begin{equation}}
\newcommand{\ee}{\end{equation}}
  
\newcommand{\beq}{\begin{eqnarray}}
\newcommand{\eeq}{\end{eqnarray}}
\newcommand{\ba}{\begin{eqnarray}}
\newcommand{\ea}{\end{eqnarray}}

\newcommand{\ua}{{\bf u}_\alpha}
\newcommand{\utwo}{\overline{\bf u}_2}


\newcommand{\ubar}{\overline{u}}
\newcommand{\vbar}{\overline{v}}
\newcommand{\phis}{{\phi}^{*}}
\newcommand{\thetas}{{\theta}^{*}}

                                   
\begin{document}

{\bf From Soulsby's book}

\section{Total Shear Stress}
Total shear stress induced by wave and currents can be calculated by (p92, (69))
\be
\tau_m = \tau_c \left[ 1+1.2\left(\frac{\tau_w}{\tau_c+\tau_w} \right)^{3.2} \right]
\ee
where $\tau_c$ and $\tau_w$ represent current and wave-induced shear stresses, respectively. 

\section{Calculating $\tau_c$ and $\tau_w$}

\subsection {$\tau_c$}
For current-induced shear stress (p53, (30))
\be
\tau_c = \rho C_D \bar{U}^2
\ee
where $\bar{U}$ is depth-averaged velocity, and $C_D$ has two forms (p48):
\be
C_D=\alpha \left( \frac{z_0}{h} \right)^\beta
\ee 
and
\be
C_D = \left[ \frac{0.40}{1+\ln(z_0/h)}  \right]
\ee
The latter one may be popular one used in sediment transport formulas.  

\subsection{$\tau_w$}

In p76, (57):
\be
\tau_w = \frac{1}{2} \rho f_w U_w^2
\ee
where $U_w$ is wave orbital velocity amplitude.

There are several formulas which can be used to calculate $f_w$
Soulsby, p78, (62):
\be
f_{w} = 1.39 \left( \frac{A}{z_0} \right)^{-0.52}
\ee
and Swart (1974) which is used for example
\ba
&& f_{wr} =0.3  \ \ \ \ \  \mbox{for} r<1.57 \\
&& f_{wr}=0.00251 \exp(5.21r^{-0.19})  \ \ \ \mbox{for} r>1.57
\ea
where $r = A/k_s$, in which $A$ is semi-orbital excursion ($U_wT/2\pi$) and $k_s = 30z_0$, Nikuradse equivalent sand grain roughness
\be
k_s =30 z_0 
\ee

 
\section{calculating roughness length $z_0$}

 $z_0$ is total roughness length
 \be
 z_0=z_{0s} + z_{0f} + z_{0t}
 \ee    
in which $z_{0s},  z_{0f}$ and  $z_{0t}$ are roughness length corresponding to skin friction, form drag, and sediment transport (mobilization). 
p48, (25)
\be
z_{0s} = d_{50}/12
\ee
For both wave/current generated ripples (p123, (90))
\be
z_{0f} = a_r \frac{\Delta_r^2}{\lambda_r}
\ee
where $\Delta_r$ and $\lambda_r$ are ripple height and length, respectively. 
For current-only case, in p59, (42)
\be
z_{0t} = \frac{5 \tau_{0s}}{30 g (\rho_s -\rho)}
\ee
where $\tau_{0s}$ is skin-friction shear stress. 

For wave generated ripples
Nielsen's formula may be used (p124, (92)), 
\be
z_{0t} = 5.67 (\theta_{ws} - 0.05)^{0.5} d_{50}
\ee
where $\theta_{ws}$ is skin-friction Shields parameter. When using Nielsen's formula, ripple height and length have to be evaluated using p122 (89)
and $a_r=0.269$. 


\end{document}  